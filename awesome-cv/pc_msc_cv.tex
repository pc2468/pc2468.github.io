%!TEX TS-program = xelatex
%!TEX encoding = UTF-8 Unicode
% Awesome CV LaTeX Template
%
% This template has been downloaded from:
% https://github.com/posquit0/Awesome-CV
%
% Author:
% Claud D. Park <posquit0.bj@gmail.com>
% http://www.posquit0.com
%
% Template license:
% CC BY-SA 4.0 (https://creativecommons.org/licenses/by-sa/4.0/)
%


%%%%%%%%%%%%%%%%%%%%%%%%%%%%%%%%%%%%%%
%     Configuration
%%%%%%%%%%%%%%%%%%%%%%%%%%%%%%%%%%%%%%
%%% Themes: Awesome-CV
\documentclass[]{awesome-cv}
\usepackage{textcomp}

%%% Override a directory location for fonts(default: 'fonts/')
\fontdir[fonts/]

%%% Configure a directory location for sections
\newcommand*{\sectiondir}{resume/}

%%% Override color
% Awesome Colors: awesome-emerald, awesome-skyblue, awesome-red, awesome-pink, awesome-orange
%                 awesome-nephritis, awesome-concrete, awesome-darknight
%% Color for highlight
% Define your custom color if you don't like awesome colors
\colorlet{awesome}{awesome-red}
\allowdisplaybreaks
%\definecolor{awesome}{HTML}{CA63A8}
%% Colors for text
%\definecolor{darktext}{HTML}{414141}
%\definecolor{text}{HTML}{414141}
%\definecolor{graytext}{HTML}{414141}
%\definecolor{lighttext}{HTML}{414141}

%%% Override a separator for social informations in header(default: ' | ')
%\headersocialsep[\quad\textbar\quad]

\hypersetup{
	colorlinks=true,
	urlcolor=cyan
}
\begin{document}
	
	%%%%%%%%%%%%%%%%%%%%%%%%%%%%%%%%%%%%%%
	%     Profile
	%%%%%%%%%%%%%%%%%%%%%%%%%%%%%%%%%%%%%%
	\begin{center}
		\headerfirstnamestyle{Prathamesh C} \headerlastnamestyle{Changde} \\
		\vspace{2mm}
		{\faGlobe\ \href{https://pc2468.github.io/}{pc2468.github.io} } | {\faEnvelope\ \href{mailto:changdeprathamesh@gmail.com}{changdeprathamesh@gmail.com}}\\ {\faGithub\ \href{https://github.com/pc2468}{github.com/pc2468}} | 
		{\faMobile\ +91 7620926132} | {\faMapMarker\ Amaravati, Maharashtra, India} 
	\end{center}
	%%%%%%%%%%%%%%%%%%%%%%%%%%%%%%%%%%%%%%
	%     Education
	%%%%%%%%%%%%%%%%%%%%%%%%%%%%%%%%%%%%%%
	\cvsection{Education}
	\begin{cventries}
		\cventry
		{M.Sc in Physics}
		{Central University of Karnataka}
		{Kalaburgi}
		{August 2024 – Current}
		{CGPA: appearing}\\
		\cventry
		{B.Sc in Physics}
		{Savitribai Phule Pune University}
		{Pune}
		{June 2020 – 2023}
		{CGPA: 8.63}\\
	\cventry
	{$12^{th}$ / HSC}
	{Maharashtra State Board}
	{Warud, Amaravati}
	{June 2019 – 2020}
	{Percentage: 68.46\%}\\
	\cventry
	{$10^{th}$ / SSC}
	{Central Board of Secondary Education}
	{Warud, Amaravati}
	{June 2017 – 2018}
	{Percentage: 66.6\%}

	\end{cventries}
	
	\vspace{-2mm}
	
	\cvsection{Skills}
\begin{cventries}
	\cventry
	{}
	{\def\arraystretch{1.15}{\begin{tabular}{ l l }
				Programing Skills:  & {\skill{ C, C\texttt{++}, Linux, Python}} \\
				Typesetting:  & {\skill{ \LaTeX}} \\
				Simulations toolkits (novice):  & {\skill{ GEANT4, GRChombo \& CosmoLattice}} \\
				Computational Physics: & {\skill { Mathematica, SciLab, LabView}} \\ Python: & {\skill{GWpy, NumPy, SciPy, Matplotlib }}  
	\end{tabular}}}
	{}
	{}
	{}
\end{cventries}
	
		\vspace{-7mm}
	\cvsection{Projects}
	\begin{cventries}
		\cventry
		{Personal endeavour to understand stellar evolution by analyzing absorption stellar spectra of known and unknown stars. Employed Python for data classification and Wein's Law for temperature calculation. Assigned spectral and luminosity classes using Morgan-Keenan system and plotted stars on HR diagram, exploring the relationship between spectral class and stellar properties Project report availabe on \href{https://github.com/pc2468/Stellar-Spectral-Classification}{Github}}
		{Classification of Stars Based on Stellar Spectra}
		{self project}
		{Sept 2021}
		{}
		
		\vspace{-5mm}
		
		\cventry
		{Worked on university project mandatory for bachelors degree to study black hole shadows, focusing on kerr black holes with scalar hair. Used ray-tracing to find hairy black hole shadows, showing smallar and exotic shapes compared to kerr black holes. Implications for EHT observations. Derived analytical Kerr solutions for ZAMO observer. Project Report available on \href{https://github.com/pc2468/Shadow-of-Kerr-Black-Holes}{Github}}
		{Shadows of Kerr black hole with and without scalar hair}
		{BSc Project}
		{May 2023}
		{}
		
		\vspace{-5mm}
		
		\cventry
		{Built a bipolar radio antenna to detect solar radio signals, using 18-gauge copper wire, coaxial cables, and an oscilloscope. Fabricated a fiber insulator with a 3D printer for structural precision. Conducted data acquisition and analysis, successfully identifying solar signals and environmental noise, enhancing understanding of radio signal behavior. The project report is available on my \href{https://github.com/pc2468?tab=repositories}{Github}}
		{Bipolar Radio Antenna}
		{Winter Holiday project}
		{Dec 2024}
		{}
		
		\vspace{-5mm}
		
		\cventry
		{Created a Python-based tool to automate the installation of Geant4, with smart configuration that adapts to the selected version. It takes care of dependencies, environment setup, and build options, making the whole process faster and easier. The script is designed to work across different systems, making it flexible and reliable. It's available on my \href{https://github.com/pc2468/Geant4}{Github} for anyone to use or contribute to.}
		{Geant4 Deployment Script}
		{Self Project}
		{Feb-March 2025}
		{}
		%\vspace{-5mm}
		
		%\cventry
		%{Investigated the nature of eternally collapsing object for the sake of consistency with the literature and studied the nature of solution thus arrived and problems with the approach.}
		%{Mathematical Inconsistencies in the Eternally Collapsing Object}
		%{}
		%{Jan 2019}
		%{}
		
	\end{cventries}
	\vspace{10mm}
	\cvsection{Conferences, Winter School and Seminar}
	\begin{cventries} 
		
		\cventry
		{Attended online winter school of Python in Research by UC SANTA CRUZ}
		{PyaR (Physics and Research)}
		{}
		{Jan 2022}
		{}
		
		\vspace{-5mm}
		\cventry
		{Attended online seminar on particle indentification in ePIC with dRICH}
		{Particle Identification in ePIC}
		{}
		{Aug 2024}
		{}
		
		\vspace{-5mm}
		\cventry 
		{Attented a talk on Generative-AI in teaching and learning of Physics in Central University of Karnataka}
		{Physics Colloquium}
		{}
		{Nov 2024}
		{}
		
		\vspace{-5mm}
		\cventry 
		{Attended the Seminar series physics on unveiling new frontiers in High Energy Physics in Central University of Karnataka}
		{High Energy Physics Seminar Series}
		{}
		{Aug-Nov 2024}
		{}
		
		
		\vspace{-5mm}
		\cventry
		{Learned about Representation Theory and Algebraic Geometry and their many interactions covering topics such as homological mirror symmetry, stability conditions, derived categories and other topics.}
		{Lagoon Webinar Series}
		{}
		{From Jan 2025}
		{}
		
		
		
		\vspace{-5mm}
		\cventry
		{Learned about various ways to use AI and ML in physics research by Brown University}
		{AI Winter School}
		{}
		{Jan 2025}
		{}
		
		\vspace{-5mm}
		\cventry
		{Attended 26 days Online Lecture series in IIRS-ISRO Distance Learning Programme}
		{ISRO START-2025}
		{}
		{Jan 2025}
		{}
		
		\vspace{-5mm}
		\cventry
		{Learned about Quantum Technologies such as Qiskit by Indian Assosition of Physics Teachers}
		{Quantum Revolution (QE aJAO-IQY2025) Workshop}
		{}
		{Feb 2025}
		{}
		
		\vspace{-5mm}
		\cventry
		{Learned about Uses of Python libraries to solve complex problems in Mathematics by IISHLS,Indus University}
		{Certification Program on Python for Mathematical Solving}
		{}
		{Feb 2025}
		{}
		
		
		\vspace{-5mm}
		\cventry
		{Learned about ideas on, cosmological correlators, phase transitions, topological defects, primordial black holes and other early structure formation.}
		{Early Universe from Home 2025 }
		{}
		{Feb 2025}
		{}
		
		
		\vspace{-5mm}
		\cventry
		{Attended sessions on Solid State, Particle, Nuclear, and Astro/Cosmological Physics. Covered topics like Quantum Hall States, Higgs Mechanism, Gravitational Waves, Neutrinoless Double-Beta Decay, and 21-cm Cosmology. Organized by Ibrahim Mirza, PhD Physics Candidate, University of Tennessee.}
		{Spring Conference 2025}
		{}
		{April 2025}
		{}
		
		\vspace{-5mm}
		\cventry
		{Attended sessions on High Energy, Nuclear, and Astrophysics. Topics included Leptogenesis, Dark Matter, Supersymmetry, Solar Neutrinos, Modified Gravity, Cosmic Dawn, and Ultra-High-Energy Neutrino Detection. Organized by Ibrahim Mirza, PhD Physics Candidate, University of Tennessee.}
		{Summer Conference on High Energy Physics \& Astrophysics 2025}
		{}
		{May 2025}
		{}
		
		\vspace{-5mm}
		\cventry
		{Participated in hands-on tutorials, quizzes, and a data challenge focused on gravitational-wave signal analysis using LIGO, Virgo, and KAGRA data. Learned to access open data, visualize waveforms, and apply matched filtering to detect binary black hole signals. Covered fundamentals of GW data recording and analysis.}
		{GW Open Data Workshop 2025}
		{}
		{May 2025}
		{}
		
		\vspace{-5mm}
		\cventry
		{Completed a four-week intensive course covering qubit fundamentals, entanglement, teleportation, quantum gates, Boolean oracles, and quantum algorithms including Deutsch, Simon’s, Grover’s, HHL, and VQE. Gained hands-on experience with quantum simulators and an introduction to Quantum Machine Learning.}
		{Quantum Computing Course – CDAC Hyderabad \& IIT Roorkee}
		{}
		{May 2025}
		{}
		
		
		\vspace{-5mm}
		\cventry
		{Attended a four-day offline school on Very High Energy Astrophysics, gaining exposure to current research on Active Galactic Nuclei, Gamma-Ray Bursts, Cosmic Rays, and Multi-Messenger Astronomy. Acquired hands-on experience with the TACTIC telescope and developed a deeper understanding of gamma-ray detection using instruments such as HESS, MAGIC, VERITAS, CTA, and the MACE telescope at Hanle, Ladakh.}
		{3$^{\text{rd}}$ DAE-BRNS School on Very High Energy Astrophysics}
		{Mount Abu, India}
		{Oct 2025}
		{}

	\end{cventries}
	%\cvsection{Future Research Interest}
	%\begin{cventries}
	%\cventry
	%{Since my research interest is very vast, it is hard for me to pindown all of them. However, some of them are listed here: $f(R)$ Gravity, Conformal Bootstrap, Black Hole Evaporation, Wormholes as gravitational solitons}
	%{}
	%{}
	%{}
	%{}
	%\end{cventries}
	
	
	
	
\end{document}